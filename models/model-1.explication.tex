\documentclass{article}
\usepackage{amsmath, amssymb}
\usepackage{enumitem}

\begin{document}

\section*{Mod\'elisation ensembliste}

\subsection*{C'est quoi une mod\'elisation ensembliste ?}

On mod\'elise le probl\`eme en utilisant des ensembles et des op\'erations sur les ensembles.

\begin{itemize}[label=\textbullet]
    \item Les variables repr\'esentent des ensembles.
    \item Les domaines sont aussi des ensembles, souvent d\'etermin\'es comme des \emph{parties} (sous-ensembles) des variables ou des \emph{parties des parties} (sous-ensembles s\'electionn\'es).
    \item Exemple : Soit $U = \{1, 2\}$ l'ensemble des variables. Les parties de $U$ ($P(U)$) sont \[ P(U) = \{ \emptyset, \{1\}, \{2\}, \{1, 2\}\}. \] Dans certains cas, on restreint le domaine \`a un sous-ensemble de $P(U)$ en fonction des contraintes.
    \item Les contraintes sont exprim\'ees avec des op\'erations ensemblistes telles que :
    \begin{itemize}[label=\textendash]
        \item L'intersection ($\cap$),
        \item L'union ($\cup$),
        \item Le cardinal d'un ensemble ($|E|$),
        \item L'inclusion ($\subseteq$).
    \end{itemize}
\end{itemize}

\subsection*{Variables $X$}

\begin{itemize}[label=\textbullet]
    \item Soit $G_{ij}$ l'ensemble des golfeurs qui jouent dans le groupe $j$ pendant la semaine $i$.
    \item $X = \{G_{ij} \mid i \in [1..w], j \in [1..g]\}$.
\end{itemize}

\subsection*{Domaines $D$}

\begin{itemize}[label=\textbullet]
    \item $D = \{X \in P(\{0, \ldots, q-1\}) \mid |X| = p\}$, \newline
          o\`u $P$ repr\'esente l'ensemble des parties (tous les sous-ensembles possibles de golfeurs).
\end{itemize}

\subsection*{Contraintes $C = C_1, C_2, C_3$}
\begin{itemize}[label=\textbullet]
   \item \textbf{$C_1$ : Chaque groupe a exactement $p$ golfeurs}
         \[ \forall i \in [1..w], \forall j \in [1..g] : |G_{ij}| = p \]
   
   \item \textbf{$C_2$ : Chaque golfeur est dans exactement un groupe par semaine}
         \[ \forall i \in [1..w], \forall k \in [1..q] : |\{j \mid k \in G_{ij}\}| = 1 \]
   
   \item \textbf{$C_3$ : Deux golfeurs ne peuvent pas se retrouver plus d'une fois ensemble pendant le tournoi}
         \[ \forall w_1, w_2 \in [1..w], w_1 \leq w_2, \forall g_1, g_2 \in [1..g], (w_1 = w_2 \rightarrow g_1 < g_2) : |G_{w_1g_1} \cap G_{w_2g_2}| \leq 1 \]
         \begin{itemize}[label=\textendash]
             \item $w_1 \leq w_2$ : Évite la redondance en ne comparant que les semaines dans un sens
             \item $(w_1 = w_2 \rightarrow g_1 < g_2)$ : Pour une même semaine, compare uniquement les groupes dans un ordre croissant
             \item $|G_{w_1g_1} \cap G_{w_2g_2}| \leq 1$ : L'intersection entre deux groupes ne doit pas dépasser un golfeur
         \end{itemize}
\end{itemize}

\subsection*{Contraintes de cassage de symétrie $C_4$, $C_5$, $C_6$}
\begin{itemize}[label=\textbullet]
\item \textbf{$C_4$ : Initialisation de la première semaine}
\[ \forall g \in [0..G-1], \forall i \in [0..P-1] : G_{0g} = G_{0g} \cup \{g*P + i\} \]
\begin{itemize}[label=\textendash]
\item Cette contrainte fixe l'affectation des golfeurs pour la première semaine
\item Exemple avec $G=2$ groupes et $P=2$ joueurs par groupe :
  \begin{itemize}
    \item Groupe 0 : $\{0, 1\}$ (car $0*2 + 0 = 0$ et $0*2 + 1 = 1$)
    \item Groupe 1 : $\{2, 3\}$ (car $1*2 + 0 = 2$ et $1*2 + 1 = 3$)
  \end{itemize}
\end{itemize}
\paragraph{Analyse formelle de la réduction de l'espace de recherche par $C_4$}

Sans la contrainte $C_4$, pour la première semaine, le nombre de façons de répartir $Q$ golfeurs en $G$ groupes de taille $P$ est donné par :

\[ \frac{Q!}{(P!)^G} \]

\textbf{\textit{Exemple simple de réduction par $C_4$ :}}

Soit $Q=4$ golfeurs $\{0,1,2,3\}$, $G=2$ groupes, et $P=2$ golfeurs par groupe.

\subsubsection*{Sans considérer l'équivalence des permutations dans les groupes}
$Q! = 4! = 24$ permutations génèrent ces arrangements :
\begin{itemize}[label=\textendash]
\item $\{\{0,1\},\{2,3\}\}$ apparaît sous 4 formes :
  \begin{itemize}
    \item $\{\{0,1\},\{2,3\}\}$
    \item $\{\{1,0\},\{2,3\}\}$
    \item $\{\{0,1\},\{3,2\}\}$
    \item $\{\{1,0\},\{3,2\}\}$
  \end{itemize}
\end{itemize}

\subsubsection*{Analyse des permutations redondantes}
Pour chaque groupe :
\begin{itemize}[label=\textendash]
\item $P! = 2!$ permutations internes ($\{0,1\}$ équivalent à $\{1,0\}$)
\item $G = 2$ groupes
\item Donc $(P!)^G = (2!)^2 = 4$ permutations représentent la même solution
\end{itemize}

\subsubsection*{Nombre réel de solutions distinctes}
\[ \frac{Q!}{(P!)^G} = \frac{24}{4} = 6 \text{ solutions uniques} \]

Cette division par $(P!)^G$ élimine le comptage multiple des solutions équivalentes dû aux permutations internes des groupes.

Avec la contrainte $C_4$ qui fixe :
\[ \forall g \in [0..G-1], \forall i \in [0..P-1] : G_{0g} = G_{0g} \cup \{g*P + i\} \]

Le nombre de possibilités pour la première semaine devient 1, car la répartition est entièrement déterminée. Cela réduit l'espace de recherche d'un facteur de $\frac{Q!}{(P!)^G}$.

Par exemple, avec $G=2$ groupes et $P=2$ joueurs par groupe ($Q=4$ golfeurs) :
\begin{itemize}[label=\textendash]
\item Sans $C_4$ : $\frac{4!}{(2!)^2} = \frac{24}{4} = 6$ configurations possibles
\item Avec $C_4$ : Une seule configuration possible, $\{\{0,1\}, \{2,3\}\}$
\end{itemize}

Cette réduction de l'espace de recherche est significative car toutes ces configurations de la première semaine sont symétriques et mènent à des solutions équivalentes.

\item \textbf{$C_5$ : Ordonnancement des groupes par minimum}
\[ \forall w \in [0..W-1], \forall g_1, g_2 \in [0..G-1], g_1 < g_2 : \min(G_{wg_1}) < \min(G_{wg_2}) \]
\begin{itemize}[label=\textendash]
\item Force un ordre croissant des minimums entre les groupes d'une même semaine
\item Exemple : avec $Q=9$, $G=3$, $P=3$ pour une semaine donnée :
  Sans $C_5$, ces configurations sont considérées comme équivalentes :
  \begin{itemize}
    \item $\{\{2,4,7\}, \{0,5,8\}, \{1,3,6\}\}$
    \item $\{\{0,5,8\}, \{1,3,6\}, \{2,4,7\}\}$
    \item $\{\{0,5,8\}, \{2,4,7\}, \{1,3,6\}\}$
    \item $\{\{1,3,6\}, \{0,5,8\}, \{2,4,7\}\}$
    \item $\{\{1,3,6\}, \{2,4,7\}, \{0,5,8\}\}$
    \item $\{\{2,4,7\}, \{1,3,6\}, \{0,5,8\}\}$
  \end{itemize}
  
  Avec $C_5$, seule cette configuration est valide :
  \[ \{\{0,5,8\}, \{1,3,6\}, \{2,4,7\}\} \]
  Car :
  \begin{itemize}
    \item $\min(\{0,5,8\}) = 0$ pour groupe 0
    \item $\min(\{1,3,6\}) = 1$ pour groupe 1
    \item $\min(\{2,4,7\}) = 2$ pour groupe 2
    \item Et $0 < 1 < 2$ respecte l'ordre croissant des minimums
  \end{itemize}
  
\item Impact sur l'espace de recherche :
  \begin{itemize}
    \item Sans $C_5$: Pour une semaine, les 3 groupes peuvent être permutés de 6 façons ($3!$)
    \item Avec $C_5$: Une seule configuration possible pour l'ordre des groupes
    \item Pour chaque semaine $w$, sans la contrainte $C_5$, les $G$ groupes peuvent être permutés de $G!$ façons différentes.
    
    \[ \text{Facteur de réduction} = G! \]
    
    \begin{itemize}[label=\textendash]
    \item Pour $G=2$ groupes : réduction par $2! = 2$
    \item Pour $G=3$ groupes : réduction par $3! = 6$
    \item Pour $G=4$ groupes : réduction par $4! = 24$
    \item Pour $G=5$ groupes : réduction par $5! = 120$
    \end{itemize}
    
Cette réduction s'applique à chaque semaine $w \in [0..W-1]$, donc l'impact total sur l'espace de recherche est de :
\[ \text{Réduction totale} = (G!)^W \]
où $W$ est le nombre de semaines.
\end{itemize}
\end{itemize}

\item \textbf{$C_6$ : Ordre croissant des maximums dans le premier groupe}
\[ \forall w_1, w_2 \in [0..W-1], w_1 < w_2 : \max(G_{w_10}) \leq \max(G_{w_20}) \]
\begin{itemize}[label=\textendash]
\item Impose que le maximum du premier groupe ($g=0$) soit croissant d'une semaine à l'autre
\item Exemple avec $Q=4$, $G=2$, $P=2$ sur deux semaines :
  Sans $C_6$, ces deux configurations sont équivalentes :
  \begin{itemize}
    \item Semaine 1, Groupe 0 : $\{1,3\}$ ($\max = 3$)
    \item Semaine 2, Groupe 0 : $\{0,2\}$ ($\max = 2$)
  \end{itemize}
  Avec $C_6$, cette configuration est invalide car $3 \not\leq 2$. \\
  Une configuration valide serait :
  \begin{itemize}
    \item Semaine 1, Groupe 0 : $\{0,2\}$ ($\max = 2$)
    \item Semaine 2, Groupe 0 : $\{1,3\}$ ($\max = 3$)
  \end{itemize}
  Cette contrainte réduit encore l'espace de recherche en ordonnant les semaines.
\end{itemize}

\paragraph{Pourquoi ces contraintes de cassage de symétrie ?}

Ces contraintes réduisent l'espace de recherche en éliminant les solutions symétriques :
\begin{itemize}[label=\textbullet]
\item $C_4$ fixe une solution initiale pour la première semaine, évitant les permutations de la configuration initiale
\item $C_5$ évite les permutations entre les groupes d'une même semaine
\item $C_6$ réduit les permutations possibles entre les semaines en imposant un ordre sur le premier groupe
\end{itemize}

\end{document}